%! suppress = TooLargeSection
%%
%% This is file `sample-lualatex.tex',
%% generated with the docstrip utility.
%%
%% The original source files were:
%%
%% samples.dtx  (with options: `sigconf')
%% 
%% IMPORTANT NOTICE:
%% 
%% For the copyright see the source file.
%% 
%% Any modified versions of this file must be renamed
%% with new filenames distinct from sample-lualatex.tex.
%% 
%% For distribution of the original source see the terms
%% for copying and modification in the file samples.dtx.
%% 
%% This generated file may be distributed as long as the
%% original source files, as listed above, are part of the
%% same distribution. (The sources need not necessarily be
%% in the same archive or directory.)
%%
%%
%% Commands for TeXCount
%TC:macro \cite [option:text,text]
%TC:macro \citep [option:text,text]
%TC:macro \citet [option:text,text]
%TC:envir table 0 1
%TC:envir table* 0 1
%TC:envir tabular [ignore] word
%TC:envir displaymath 0 word
%TC:envir math 0 word
%TC:envir comment 0 0
%%
%%
%% The first command in your LaTeX source must be the \documentclass
%% command.
%%
%% For submission and review of your manuscript please change the
%% command to \documentclass[manuscript, screen, review]{acmart}.
%%
%% When submitting camera ready or to TAPS, please change the command
%% to \documentclass[sigconf]{acmart} or whichever template is required
%% for your publication.
%%
%%
\documentclass[sigconf]{acmart}
\usepackage{graphicx}
\usepackage{geometry}
\usepackage{hyperref}
\usepackage{color}
\hypersetup{colorlinks=true,linkcolor=blue}
\geometry{a4paper}
%\citestyle{acmauthoryear}

%%
%% \BibTeX command to typeset BibTeX logo in the docs
\AtBeginDocument{%
  \providecommand\BibTeX{{%
    \normalfont B\kern-0.5em{\scshape i\kern-0.25em b}\kern-0.8em\TeX}}}

%% Rights management information.  This information is sent to you
%% when you complete the rights form.  These commands have SAMPLE
%% values in them; it is your responsibility as an author to replace
%% the commands and values with those provided to you when you
%% complete the rights form.
\setcopyright{rightsretained}
\copyrightyear{2022}
%\acmYear{2018}
\acmDOI{}

%% These commands are for a PROCEEDINGS abstract or paper.
\acmConference[Information Society 2021]{Information Society 2021: 24th international multiconference}{4--8 October 2021}{Ljubljana, Slovenia}
%%
%%  Uncomment \acmBooktitle if the title of the proceedings is different
%%  from ``Proceedings of ...''!
%%
%%\acmBooktitle{Woodstock '18: ACM Symposium on Neural Gaze Detection,
%%  June 03--05, 2018, Woodstock, NY}
\acmPrice{}
\acmISBN{}
\settopmatter{printccs=false, printacmref=false}


%%
%% Submission ID.
%% Use this when submitting an article to a sponsored event. You'll
%% receive a unique submission ID from the organizers
%% of the event, and this ID should be used as the parameter to this command.
%%\acmSubmissionID{123-A56-BU3}

%%
%% For managing citations, it is recommended to use bibliography
%% files in BibTeX format.
%%
%% You can then either use BibTeX with the ACM-Reference-Format style,
%% or BibLaTeX with the acmnumeric or acmauthoryear sytles, that include
%% support for advanced citation of software artefact from the
%% biblatex-software package, also separately available on CTAN.
%%
%% Look at the sample-*-biblatex.tex files for templates showcasing
%% the biblatex styles.
%%

%%
%% The majority of ACM publications use numbered citations and
%% references.  The command \citestyle{authoryear} switches to the
%% "author year" style.
%%
%% If you are preparing content for an event
%% sponsored by ACM SIGGRAPH, you must use the "author year" style of
%% citations and references.
%% Uncommenting
%% the next command will enable that style.
%%\citestyle{acmauthoryear}


%%
%% end of the preamble, start of the body of the document source.
\begin{document}

%%
%% The "title" command has an optional parameter,
%% allowing the author to define a "short title" to be used in page headers.
\title{Analysis of Transfer Learning for Named Entity Recognition in South-Slavic Languages}

%%
%% The "author" command and its associated commands are used to define
%% the authors and their affiliations.
%% Of note is the shared affiliation of the first two authors, and the
%% "authornote" and "authornotemark" commands
%% used to denote shared contribution to the research.
\author{Nikola Ivačič}
%\authornote{Both authors contributed equally to this research.}
\email{nikola.ivacic@gmail.com}
%\orcid{1234-5678-9012}
%\author{G.K.M. Tobin}
%\authornotemark[1]
%\email{webmaster@marysville-ohio.com}
\affiliation{%
  \institution{Jožef Stefan Institute\\
  Jamova cesta 39}
  \streetaddress{Jamova cesta 39}
  \city{Ljubljana}
  \country{Slovenia}
  \postcode{1000}
}

\author{Hanh Tran}
\affiliation{%
  \institution{Jožef Stefan Institute\\
  Jamova cesta 39}
  \streetaddress{Jamova cesta 39}
  \city{Ljubljana}
  \country{Slovenia}
  \postcode{1000}}

\author{Boshko Koloski}
\affiliation{%
  \institution{Jožef Stefan Institute\\
  Jamova cesta 39}
  \streetaddress{Jamova cesta 39}
  \city{Ljubljana}
  \country{Slovenia}
  \postcode{1000}
}

\author{Senja Pollak}
\affiliation{%
 \institution{Jožef Stefan Institute\\
  Jamova cesta 39}
  \streetaddress{Jamova cesta 39}
  \city{Ljubljana}
  \country{Slovenia}
  \postcode{1000}}

\author{Marko Pranjić}
\affiliation{%
  \institution{Jožef Stefan Institute\\
  Jamova cesta 39}
  \streetaddress{Jamova cesta 39}
  \city{Ljubljana}
  \country{Slovenia}
  \postcode{1000}}

\author{Matthew Purver}
\affiliation{%
  \institution{School of Electronic Engineering and Computer Science\\
  Queen Mary University of London}
  \streetaddress{Mile End Road\\
   London E1 4NS, UK}
  \city{London}
  \country{UK}}

\author{Nada Lavrač}
\affiliation{%
  \institution{Jožef Stefan Institute\\
  Jamova cesta 39}
  \streetaddress{Jamova cesta 39}
  \city{Ljubljana}
  \country{Slovenia}
  \postcode{1000}}

%%
%% By default, the full list of authors will be used in the page
%% headers. Often, this list is too long, and will overlap
%% other information printed in the page headers. This command allows
%% the author to define a more concise list
%% of authors' names for this purpose.
\renewcommand{\shortauthors}{Ivačič et al.}

%%
%% The abstract is a short summary of the work to be presented in the
%% article.
\begin{abstract}
  This paper presents analysis of a Named Entity Recognition task for South-Slavic languages using the pre-trained multilingual neural network models.
  Observing the performance metrics from prior research showed that the performance of the fine-tuned multilingual neural model is very close to the performance of the monolingual one.
  This observation lead us to a question that this paper aims to answer: Can the fine-tuning of a multilingual pre-trained embeddings, with other than the target language corpora, improve named entity recognition for a specific language?
\end{abstract}

%%
%% Keywords. The author(s) should pick words that accurately describe
%% the work being presented. Separate the keywords with commas.
\keywords{natural language processing, named entity recognition, neural networks, text classification, multilingual pre-trained models}
%% A "teaser" image appears between the author and affiliation
%% information and the body of the document, and typically spans the
%% page.
%% NikComm\begin{teaserfigure}
%% NikComm  \includegraphics[width=\textwidth]{sampleteaser}
%% NikComm  \caption{Seattle Mariners at Spring Training, 2010.}
%% NikComm  \Description{Enjoying the baseball game from the third-base
%% NikComm  seats. Ichiro Suzuki preparing to bat.}
%% NikComm  \label{fig:teaser}
%% NikComm\end{teaserfigure}


%%
%% This command processes the author and affiliation and title
%% information and builds the first part of the formatted document.
\maketitle

\section{Introduction}
\label{sec:introduction}
Multilingual neural models simplify scalability and applications where many languages are needed for the task.
Getting the performance of a multilingual neural model as close as possible to the performance of a monolinugual one can be very beneficial.
Named Entity Recognition (NER) is also one of the corner stones of the Natural Language Processing (NPL) tasks in many text processing systems.

We observed two prior works:
\begin{itemize}
\item BSNLP: 3rd Shared Task on SlavNER\cite{piskorski-etal-2021-slav} submitted paper by Prelevikj and Žitnik et al.\cite{prelevikj-zitnik-2021-multilingual}
\item Clipping industry project developed by Department of Knowledge Technologies at Jožef Stefan Institute\cite{KTIJS}
\end{itemize}
Prior work has shown that the monolingual NER model performance for Slovene language is practically equal to the performance of a multilingual one.
Additionally the Clipping industry project multilingual model was fine-tuned with other language corpora from the same - South-Slavic family.

The question that naturally arises from prior work and this paper tries to answer is this: Did the fine-tuning with related languages influence the performance of a multilingual model?
The problem with comparison of both prior work is that the:
\begin{itemize}
\item corpus preprocessing
\item pre-trained models
\item hyper-parameters
\item performance measures
\end{itemize}
were not strictly the same in both experiments so one has to be really cautious when evaluating the performance metrics, or deriving conclusions from it.

\section{Data Description}
\label{sec:data-description}
We used the most common and established NER corpora for selected languages where possible (see Table~\ref{tab:corpora}).
The assumption and strategy with gathering corpora was also: ``the more the better''.
This would most probably bring us closer to the truth.
\begin{table}[H]
  \caption{List of Used Corpora}
  \label{tab:corpora}
  \begin{tabular}{llrrl}
    \toprule
    Lang.&Abbr.&Sentences&Tokens&Name\\
    \midrule
    sl&bsnlp&18106&400291&BSNLP 2017/21\cite{piskorski-etal-2021-slav}\\
    sl&500k&9483&193611&ssj500k 2.3\cite{ssj500k-23}\\
    sl&ewsd&2024&31233&{ELEXIS}-{WSD} 1.0\cite{ELEXIS-WSD-10}\\
    sl&scr&18139&391526&{SentiCoref} 1.0\cite{SentiCoref-10}\\
    \midrule
    hr&bsnlp&820&18704&BSNLP 2017 and 2021\cite{piskorski-etal-2021-slav}\\
    hr&500k&24780&504227&hr500k 1.0\cite{hr500k-10}\\
    \midrule
    sr&set&3891&86726&{SETimes}.{SR} 1.0\cite{SETimes-SR-1.0}\\
    \midrule
    bs&wann&8917&199378&WikiANN - PAN-X\cite{rahimi-etal-2019-massively}\\
    \midrule
    mk&wann&16227&156467&WikiANN - PAN-X\cite{rahimi-etal-2019-massively}\\
  \bottomrule
  \end{tabular}
\end{table}
 We used NER tags in IOB2\cite{IOB2} format from CoNLL-2003 shared task\cite{CoNLL2003} as a common denominator for all corpora and experiments.
Each corpus was first combined if split, then converted to a common format, reshuffled, and split to train / dev / test set in a 80 / 10 / 10 ratio.

When the combined corpora was used, we concatenated the sets without further reshuffling so that the experiments can be repeated.

The Slovene corpora was obtained from BSNLP and parts of a newly published combined Training corpus {SUK} 1.0\cite{SUK-1.0}
%Prepared and converted data is available on Github for convenience and possible manual review (move to online resources?).

\subsection{Data Conversion}
\label{subsec:data-conversion}
First obstacle was different NER tags used in a corpora.
Decision was made to keep only common tags: PER, LOC, ORG\@.
For example: BSNLP corpus uses PRO and EVT tags, while WikiANN corpus is missing a MISC tag that is common to ssj500k and hr500k training corpora.
All non-common tags, including MISC, were replaced with O (outside IOB) tag.

Second obstacle was the difference in format.
BSNLP corpus, for instance, uses separate files for verbatim text and NER tags, with no positional reference between one another.
To solve this problem we used CLASSLA\cite{ljubesic-dobrovoljc-2019-neural} sentence segmentation and tokenization with custom conversion script.

The third step was to remove sentences longer than 128 tokens, and convert corpora from common CoNLL format to CSV format with two fields:
\begin{itemize}
  \item sentence: whitespace separated sentence word tokens.
  \item ner: white space separated NER tags for each sentence word token.
\end{itemize}

The fourth and final step was to split the corpus data to train, test and dev sets.

\subsection{Corpora analysis}
\label{subsec:corpora-analysis}
Comparing the corpora, showed the differences that could potentially be problematic for obtaining aligned model performance.
Especially considering the NER tag ratios where the WikiANN automatically annotated corpora structure was standing
out (see Table~\ref{tab:corpora_analysis} and Figure~\ref{fig:corpora_analysis}).
\begin{table}[H]
  \caption{Analysis of Combined Corpora}
  \label{tab:corpora_analysis}
  \begin{tabular}{lrrrrr}
    \toprule
    lang&tok.\ / sent.&NER / tok.&PER&LOC&ORG\\
    \midrule
    sl&21.29&9.09\%&31.70\%&22.20\%&34.13\%\\
    hr&20.43&7.41\%&28.71\%&20.55\%&30.82\%\\
    sr&22.29&12.01\%&29.96\%&30.12\%&32.35\%\\
    bs&\textbf{7.81}&\textbf{36.91\%}&31.65\%&29.67\%&38.67\%\\
    mk&\textbf{9.64}&\textbf{28.07\%}&34.89\%&30.32\%&34.79\%\\
    \bottomrule
    \multicolumn{6}{c}{PER, LOC and ORG are ratios with respect to NER}
    %\multicolumn{6}{r}{MISC tag is excluded}
  \end{tabular}
\end{table}

\begin{figure}[h]
  \caption{WikiANN corpus skew}
  \label{fig:corpora_analysis}
  \centering
  \includegraphics[width=\linewidth]{wikiann-skew}
\end{figure}

Fortunately, we were unable to detect any inconsistencies regarding performance measurements.

\section{Methods}
\label{sec:methods}

\subsection{Measurements}
\label{subsec:measurements}
For the classification measurements the Seqeval library\cite{seqeval} was used.
Although library uses CoNLL evaluation by default, we rather chose ``strict'' mode evaluation.
Strict mode takes into account also the ``beginning'' and ``inside'' part of the IOB when calculating measurements.
Therefore the NER tags must match exactly.

\subsection{Model selection}
\label{subsec:model-selection}
For all experiments Hugging Face's transformers Python library\cite{wolf-etal-2020-transformers} was used.
Prior work used two pre-trained multilingual models from Hugging Face, among other, which produced best results:
\begin{itemize}
  \item BERT multilingual base model (cased)\cite{DBLP:journals/corr/abs-1810-04805}
  \item XLM-RoBERTa (base sized model)\cite{DBLP:journals/corr/abs-1911-02116}
\end{itemize}

Due to the task and resource limitations, we continued with our experiments using only XLM-RoBERTa base sized model,
since the initial baseline tests showed, that XLM-RoBERTa consistently outperforms BERT model (see Table ~\ref{tab:roberta_vs_bert})

\begin{table}[H]
  \caption{XLM-RoBERTa vs. BERT multilingual cased}
  \label{tab:roberta_vs_bert}
  \begin{tabular}{llrrrr}
    \toprule
    Model&Corpora&Precision&Recall&F1&Accuracy\\
    \midrule
    mcbert&bsnlp&0.935&0.931&0.933&0.990\\
    xlmrb&bsnlp&0.947&0.954&\textbf{0.951}&0.993\\
    \midrule
    mcbert&ssj500k&0.826&0.828&0.827&0.984\\
    xlmrb&ssj500k&0.869&0.902&\textbf{0.885}&0.990\\
    \midrule
    mcbert&bsnlp+ssj500k&0.898&0.903&0.900&0.986\\
    xlmrb&bsnlp+ssj500k&0.929&0.931&\textbf{0.930}&0.990\\
    \bottomrule
  \end{tabular}
\end{table}

\subsection{Parameters}
\label{subsec:parameters}
For all the experiments the following hyper-parameters were used:
\begin{itemize}
  \item 256 max-length for tokenizer
  \item PyTorch's AdamW algorithm with 5e-5 learning rate
  \item batch size of 20
  \item 40 epochs
  \item F1 score for best model selection nad training progression.
\end{itemize}

\subsection{Method}
\label{subsec:method}
The selected method was first to train the baseline model for each language, and produce NER classification measurements.
Baseline models were fine-tuned with only one - target language.
Next, we combined additional language corpora, re-train the model, and again measure performance against the target language test set.

%Czech language was also used, because it is not as closely related as other languages to Slovene and could give us additional insights in possible performance changes.
\section{Evaluation}
\label{sec:evaluation}
For the evaluation we used target language test set, and measure performance against the fine-tuned models.

Overall F1 score, used in the evaluation tables and figures, is a macro averaged F1 score over all three NER tags.

We excluded Bosnian and Macedonian language analysis, due to time and resource constraints.
Their corpora structure also influenced this decision, but nevertheless, we included them in the evaluation of other languages, to see if something interesting would emerge.

Average distance from the baseline was used as a measure, to show overall variability of different models tested wih same test set.
\subsection{Slovene}
\label{subsec:slovene}

\begin{figure}[H]
  \caption{Slovene language test set model performance}
  \label{fig:eval_sl}
  \centering
  \includegraphics[width=\linewidth]{eval_sl}
\end{figure}

The Slovene test set shows surprising model stability.
This stability comes, assumingly, from a larger corpora compared to the others.
It might be that the quality of the corpora also plays a crucial role in this observation.

\begin{table}[H]
  \caption{Slovene language test set model performance}
  \label{tab:eval_sl}
  \begin{tabular}{lrrrrr}
    \toprule
    Model&PER F1&LOC F1&ORG F1&Overall F1\\
    \midrule
    baseline sl&0.963&0.963&0.931&0.952\\
    \midrule
    sl.sr&0.963&0.955&0.921&0.946\\
    sl.hr&0.962&0.96&0.924&0.948\\
    sl.hr.sr&0.964&0.958&0.925&0.949\\
    sl.hr.sr.bs&0.964&0.953&0.926&0.948\\
    sl.hr.sr.bs.mk&0.962&0.952&0.926&0.947\\
    \midrule
    avg.\ dist.&0.00071&0.00695&0.00634&0.00433\\
    \bottomrule
  \end{tabular}
\end{table}

If we observe the average distance from the baseline, in the table's last row, we can see that it is only near 0.5\%.

\subsection{Croatian}
\label{subsec:croatian}

Croatian language test set shows greater variability when tested with different models.
The greatest variability can be observed on the ORG tag.
It might be that the other corpora training is influencing variability.
We can see that the average distance from the baseline is between 0.5\% and 1\%.
\begin{figure}[h]
  \caption{Croatian language test set model performance}
  \label{fig:eval_hr}
  \centering
  \includegraphics[width=\linewidth]{eval_hr}
\end{figure}

\begin{table}[H]
  \caption{Croatian language test set model performance}
  \label{tab:eval_hr}
  \begin{tabular}{llrrrr}
    \toprule
    Model&PER F1&LOC F1&ORG F1&Overall F1\\
    \midrule
    baseline hr&0.934&0.911&0.874&0.906\\
    \midrule
    hr.sr&0.932&0.921&0.888&0.914\\
    sl.hr&0.925&0.915&0.878&0.906\\
    hr.sr.bs&0.922&0.912&0.856&0.897\\
    sl.hr.sr&0.923&0.908&0.865&0.899\\
    sl.hr.sr.bs&0.938&0.927&0.873&0.912\\
    sl.hr.sr.bs.mk&0.925&0.911&0.861&0.899\\
    \midrule
    avg.\ dist.&0.00757&0.00554&0.00977&0.00624\\
    \bottomrule
  \end{tabular}
\end{table}

\subsection{Serbian}
\label{subsec:serbian}
Serbian language test set is the only one that increased its performance measurements with respect to its baseline.
Its average distance in performance measurements from the baseline is from approximately 0.5\% to 2.5\%.
The main suspect for this phenomena is the Serbian corpora size.
It's the smallest included in this analysis, and as such, the most easily influenced by other corpora.
\begin{figure}[h]
  \caption{Serbian language test set model performance}
  \label{fig:eval_sr}
  \centering
  \includegraphics[width=\linewidth]{eval_sr}
\end{figure}

\begin{table}[H]
  \caption{Serbian language test set model performance}
  \label{tab:eval_sr}
  \begin{tabular}{llrrrr}
    \toprule
    Model&PER F1&LOC F1&ORG F1&Overall F1\\
    \midrule
    baseline sr&0.962&0.979&0.914&0.954\\
    \midrule
    sl.sr&0.979&0.98&0.934&0.965\\
    hr.sr&0.987&0.988&0.956&0.978\\
    hr.sr.bs&0.982&0.987&0.945&0.973\\
    sl.hr.sr&0.979&0.979&0.946&0.969\\
    sl.hr.sr.bs&0.971&0.976&0.92&0.957\\
    sl.hr.sr.bs.mk&0.988&0.978&0.942&0.97\\
    \midrule
    avg.\ dist.&0.01885&0.00372&0.02603&0.01504\\
    \bottomrule
  \end{tabular}
\end{table}

\section{Conclusion}
\label{sec:conclusion}
We have shown that the model performance is not influenced substantially when trained with other than a target language.
Especially when target language is presented with enough training data.
Fine-tuning with other languages may be beneficial only for the ``low resource'' languages.

Our initial assumption doesn't hold, but the result is stil positive when model needs to be used in a multilingual setting.
When using multilingual model in an application, we can be quite sure that the performance would not degrade much if we would to fine-tune the model with additional language corpora.

%%
%% The acknowledgments section is defined using the "acks" environment
%% (and NOT an unnumbered section). This ensures the proper
%% identification of the section in the article metadata, and the
%% consistent spelling of the heading.
%\begin{acks}
%To Robert, for the bagels and explaining CMYK and color spaces.
%\end{acks}

%%
%% The next two lines define the bibliography style to be used, and
%% the bibliography file.
\bibliographystyle{ACM-Reference-Format}
\bibliography{atlner}

%%
%% If your work has an appendix, this is the place to put it.
\appendix

\section{Online Resources}

All the measurement results, including the ones not presented in this paper, together with code and corpora, can be found online at
{\color{blue}\href{https://github.com/ivlcic/trans-ner}{Github}}.
A more user friendly results are also available as a
{\color{blue}\href{https://docs.google.com/spreadsheets/d/16UaPR-qneNN8mlzjNkjc9kdry2eQl4J84t3EfDOtBj0/edit?usp=sharing}{Spreadsheet}}.

\end{document}
\endinput
%%
%% End of file `sample-lualatex.tex'.
